\documentclass{article}
\usepackage[utf8]{inputenc}
\usepackage{multicol}

\title{Exam 1 Notes Page}
\author{}
\date{}

\begin{document}

\begin{multicols}{3}

\_\_eq\_\_ compares addresses of two items by default. Overrides ==.\\\\
\_\_getitem\_\_ gets the item at a specific key
\_\_init\_\_ is Python's constructor, first param is always passed in and is usually called "self"
\_\_iter\_\_ returns the iterable for a Python object
\_\_next\_\_ pops the next item off of an iterable
assert is good for pre and postconditions, bad for user input or testing
unittest is good for testing
use try and except with StopIteration exception to build a custom iterator
Mutable: List, Set
Immutable: Tuple, FrozenSet
floordiv() is like Java's integer division
def declares a function
issubclass with parameter classname checks if a class is a subclass of another class
truediv() is like Java's double division
pass is used when you have no code to put inside an if/else/try/except
Use a comma at the end of a single item tuple
Yield is used as a return inside generators
Map applies a function to all items inside of an iterable. The first parameter is the function to be applied, the second is the list that is iterable.
Reduce takes a two param function and a list, and goes iterating through with a "cur" and a "next" as the first and second parameter being passed into the function.
For reduce, the default is either the first value or the third parameter
Positional arguments have to go before named arguments
For parameters, you basically replace * with all the values as positional arguments, and replace ** (dict values) with named arguments
Passing in the same parameter twice into a function causes an error
Using * then ** in the function declaration can let the user pass anything into a function

\end{multicols}

\end{document}
